\documentclass[]{article}
\usepackage{lmodern}
\usepackage{amssymb,amsmath}
\usepackage{ifxetex,ifluatex}
\usepackage{fixltx2e} % provides \textsubscript
\ifnum 0\ifxetex 1\fi\ifluatex 1\fi=0 % if pdftex
  \usepackage[T1]{fontenc}
  \usepackage[utf8]{inputenc}
\else % if luatex or xelatex
  \ifxetex
    \usepackage{mathspec}
  \else
    \usepackage{fontspec}
  \fi
  \defaultfontfeatures{Ligatures=TeX,Scale=MatchLowercase}
\fi
% use upquote if available, for straight quotes in verbatim environments
\IfFileExists{upquote.sty}{\usepackage{upquote}}{}
% use microtype if available
\IfFileExists{microtype.sty}{%
\usepackage{microtype}
\UseMicrotypeSet[protrusion]{basicmath} % disable protrusion for tt fonts
}{}
\usepackage[margin=1in]{geometry}
\usepackage{hyperref}
\hypersetup{unicode=true,
            pdftitle={Marno Verbeek-A Guide to Modern Econometrics (2004) Unemplyment Benefit pg 198},
            pdfborder={0 0 0},
            breaklinks=true}
\urlstyle{same}  % don't use monospace font for urls
\usepackage{graphicx,grffile}
\makeatletter
\def\maxwidth{\ifdim\Gin@nat@width>\linewidth\linewidth\else\Gin@nat@width\fi}
\def\maxheight{\ifdim\Gin@nat@height>\textheight\textheight\else\Gin@nat@height\fi}
\makeatother
% Scale images if necessary, so that they will not overflow the page
% margins by default, and it is still possible to overwrite the defaults
% using explicit options in \includegraphics[width, height, ...]{}
\setkeys{Gin}{width=\maxwidth,height=\maxheight,keepaspectratio}
\IfFileExists{parskip.sty}{%
\usepackage{parskip}
}{% else
\setlength{\parindent}{0pt}
\setlength{\parskip}{6pt plus 2pt minus 1pt}
}
\setlength{\emergencystretch}{3em}  % prevent overfull lines
\providecommand{\tightlist}{%
  \setlength{\itemsep}{0pt}\setlength{\parskip}{0pt}}
\setcounter{secnumdepth}{0}
% Redefines (sub)paragraphs to behave more like sections
\ifx\paragraph\undefined\else
\let\oldparagraph\paragraph
\renewcommand{\paragraph}[1]{\oldparagraph{#1}\mbox{}}
\fi
\ifx\subparagraph\undefined\else
\let\oldsubparagraph\subparagraph
\renewcommand{\subparagraph}[1]{\oldsubparagraph{#1}\mbox{}}
\fi

%%% Use protect on footnotes to avoid problems with footnotes in titles
\let\rmarkdownfootnote\footnote%
\def\footnote{\protect\rmarkdownfootnote}

%%% Change title format to be more compact
\usepackage{titling}

% Create subtitle command for use in maketitle
\providecommand{\subtitle}[1]{
  \posttitle{
    \begin{center}\large#1\end{center}
    }
}

\setlength{\droptitle}{-2em}

  \title{Marno Verbeek-A Guide to Modern Econometrics (2004) Unemplyment Benefit
pg 198}
    \pretitle{\vspace{\droptitle}\centering\huge}
  \posttitle{\par}
    \author{}
    \preauthor{}\postauthor{}
    \date{}
    \predate{}\postdate{}
  

\begin{document}
\maketitle

\subsection{Marno Verbeek-A Guide to Modern Econometrics (2004)
Unemplyment Benefit
pg}\label{marno-verbeek-a-guide-to-modern-econometrics-2004-unemplyment-benefit-pg}

\subsection{Logit Probit Regression - Marno Verbeek-A Guide to Modern
Econometrics (2004) pg
198}\label{logit-probit-regression---marno-verbeek-a-guide-to-modern-econometrics-2004-pg-198}

\subsubsection{7.1.6 Illustration: the Impact of Unemployment Benefits
on
Recipiency}\label{illustration-the-impact-of-unemployment-benefits-on-recipiency}

As an illustration we consider a sample3 of 4877 blue collar workers who
lost their jobs in the US between 1982 and 1991, taken from a study by
McCall (1995). Not all unemployed workers eligible for unemployment
insurance (UI) benefits apply for it, probably due to the associated
pecuniary and psychological costs. The percentage of eligible unemployed
blue collar workers that actually applies for UI benefits is called the
take-up rate, and it was only 68\% in the available sample. It is
therefore interesting to investigate what makes people decide not to
apply.

The amount of UI benefits a person can receive depends upon the state of
residence, the year of becoming unemployed, and his or her previous
earnings. The replacement rate, defined as the ratio of weekly UI
benefits to previous weekly earnings, varies from 33\% to 54\% with a
sample average of 44\%, and is potentially an important factor for an
unemployed worker's choice to apply for unemployment benefits. Of
course, other variables may influence the take-up rate as well. Due to
personal characteristics, some people are more able than others to find
a new job in a short period of time and will therefore not apply for UI
benefits. Indicators of such personal characteristics are schooling,
age, and, due to potential (positive or negative) discrimination in the
labour market, racial and gender dummies. In addition, preferences and
budgetary reasons, as reflected in the family situation, may be of
importance. Due to the important differences in the state unemployment
rates, the probability of finding a new job varies across states and we
will therefore include the state unemployment rate in the analysis. The
last type of variables that could be relevant relates to the reason why
the job was lost. In the analysis we will include dummy variables for
the reasons: slack work, position abolished, and end of seasonal work.

We estimate three different models, the results of which are presented
in Table 7.2. The \emph{linear probability model is estimated by
ordinary least squares}, so no corrections \textbf{for
heteroskedasticity are made and no attempt is made to keep the implied
probabilities between 0 and 1}. The \textbf{logit} and \textbf{probit
model} are both estimated by maximum likelihood. Because the logistic
distribution has a variance of ??\^{}2/3, the estimates of ?? obtained
from the logit model are roughly a factor ??/???3 larger than those
obtained from the probit model, acknowledging the small differences in
the shape of the distributions. Similarly, the estimates for the linear
probability model are quite different in magnitude and approximately
four times as small as those for the logit model (except for the
intercept term). Looking at the results in Table 7.2, we see that the
signs of the coefficients are identical across the different
specifications, while the statistical significance of the explanatory
variables is also comparable. This is not an unusual finding.
Qualitatively, the different models typically do not provide different
answers.

\section{Set working directory:}\label{set-working-directory}

\begin{verbatim}
setwd("C:/Users/Willians/Desktop/Mestrado/Econometria - Schonerwaldt")  # note / instead of \ in windows 
\end{verbatim}

\section{Store the data from benefits.dta in an object
(database):}\label{store-the-data-from-benefits.dta-in-an-object-database}

\begin{verbatim}
database <- read.dta("C:/Users/Willians/Desktop/Mestrado/Econometria - Schonerwaldt/R Scripts/Data/unemployment_benefits/benefits.dta",  convert.factors=FALSE)
\end{verbatim}

\section{Explore the data features (columns) and 5 Number Summary
Stats}\label{explore-the-data-features-columns-and-5-number-summary-stats}

\begin{verbatim}
head(database)
summary(database)
\end{verbatim}

\section{Create a vector with the packages to be
installed:}\label{create-a-vector-with-the-packages-to-be-installed}

\begin{verbatim}
list.of.packages <- c("AER", "sandwich", "lmtest", "car", "dplyr", "stargazer", "ggplot2", "foreign",
                      "openintro","OIdata", "gdata", "doBy","ivpack", "psych","plm", "readxl")

new.packages <- list.of.packages[!(list.of.packages %in% installed.packages()[,"Package"])]

if(length(new.packages)) install.packages(new.packages)

lapply(list.of.packages, require, character.only = TRUE)
\end{verbatim}

\section{Logit Regression}\label{logit-regression}

\begin{verbatim}
logit <- glm(y ~ rr + rr2 + age + age2 + tenure + slack + abol + seasonal + head + married + dkids + dykids + smsa + nwhite + yrdispl + school12 + male + statemb + stateur, family = binomial(link = "logit"), 
             data = database)
summary (logit)
\end{verbatim}

\section{Probit Regression}\label{probit-regression}

\begin{verbatim}
probit <- glm(y ~ rr + rr2 + age + age2 + tenure + slack + abol + seasonal + head + married + dkids + dykids + smsa + nwhite + yrdispl + school12 + male + statemb + stateur, family = binomial(link = "probit"), 
              data = database)
summary (probit)
\end{verbatim}

\section{LPM -\textgreater{} MQO Normal Linear
Regression}\label{lpm---mqo-normal-linear-regression}

\begin{verbatim}
lpm <- lm(y ~ rr + rr2 + age + age2 + tenure + slack + abol + seasonal + head + married + dkids + dykids + smsa + nwhite + yrdispl + school12 + male + statemb + stateur, data = database)
summary(lpm)
\end{verbatim}


\end{document}
